%This is a scrap LaTeX document. It has no header. This is mainly due to the fact that I don't feel like writing one.

%Commands:

%gr = GRACE
%tek = TEKNIQ
%kev = KEVIN
%ker = KERI

\gr Alright class, I gotta talk to you about things that are hard to translate.

\gr Ceasium-133 does\ldots{} somthing nine million times a seccond. You're civilization has atomic clocks, and you define your\ldots{} time thing same as we do.

\gr A meter is one three hundraid millionth of a light seccond. Its impossible to mesure the speed of light but you guys have complex optics (we do too but I don't remember how it works) and can take nanometers and scale them up anthro size with lazers and stuff. I gave you a meter stick, you mesured it, and your definition of a meter is within a tenth of a percent of the real deal and thats good enough for me.

\gr That small percent difference adds up fast, so for large things we use the average distance from Earth to Sol as a unit of mesurement, wich you guys have masured acuratly.

\tek It's not enough for basic stuff. We are still missing a lot of the system though.

\gr I think there's four other units but one of them is way more importent.

\tek Yeah, the kilokam.

\gr Kilogram.

\kev I call it $\mathbb{massthing}$ %Shows up as random math symbols representing the alian language.

\ker Shut up!

I translate his name as Kevin because he absolutly acts like one.

\noindent$\mathbb{randombro}$: Hey, is the dome we're in 37 meters or am AI doing it wrong?

\gr GUYS! I wanna talk about killograms!

Everyone stares blankly.

\gr You guys arnt going to like this but-- Just a sec

I check my notes. Dang, this is worse than I thought.

\gr \emph{sigh} Ok, uh, A photon, a radio particle, that ocilates $h^{-1}$ times per seccond has enough energy to accelerate $1kg$ at $1m/s^2$ over the course of $1m$

\kev The heck is that su---

K'eri slaps Kevin because I'm not alloued to.

\gr I said you aint gonna like it! Anyway, you know what $1m/s^2$ would mean, right?

\kev Whats a square seccond in square \emph{reijet}s (Mosphi'ite unit of \emph{length})

\tek Acceleration. Speed go up. Or down. Dosn't matter.

\kev Wait, a seccond mesures time?

\ker Bros got 16KiB of ram.

\gr Literally?

\ker Nah, but his AI is\ldots{} undeveloped.

\kev I mean, I could learn but my machine learning system is $\mathbb{loadabs}$ (Probibly a swear) and makes me go to 784 $\mathbb{notkalvin}$

\tek Grace, thats a temperture unit and it is greater than the boiling point of water by the way.

\kev It makes everyone except me very uncomfortable.

\ker Yeah, you'd boil an organic being alive!

\gr Man, you'd thing a civilization ran by computers would at least have water cooling or somthing--

\tek I use water cooling. The kids don't. It would boil. Noew I'm interested in temperture units.

\gr WE GET THERE WHEN WE GET THERE.

\gr Ok, acceleration. If you want to accelerate an object with mass, what is that called?

\ker Pettal to the metal?

\kev Can I run my machine learning? I promice not to boil anyone.

\tek \emph{sigh} Its $\mathbb{fortis}$ isn't it? Basic stuff.

I edit the translator so whatever TEKNIQ just said becomes 'force`



\gr One newton is accelerating a kilogram at $1m/s^2$. Its force, not energy.

\tek Great. Now you say somthing like ''this thing has a force of X newtons`` and we can calculate a kilogram!

\gr We're not there yet. We need to talk about potential energy.\

\ker And when you do a force over time, thats energy!

\gr No, its---

\tek No no, he's right in his home country. Our equivolent of the metric system does what I think you're about to talk about.

\gr Force, applied over a distance. We call it energy.

\gr Well, newtons per seccond cancels out to killogram meters per seccond so a killogram moving at 1m/s would have 1 ''energy``

\tek Kenetic energy I guess?

\gr Anyway, energy, not kenetic energy, just energy in general is defined by pushing things. When you use $1N$ of force to push somthing $1m$ thats called a Joule.

\ker Hah, I told you energy was green!

\kev No! Its blue! I saw electricity and its blue!

\gr Wha?

\tek Remember, you taught kepler's laws in Kerbal Space Program. They think you're talking about 'Jool` the planet. 

\gr JOULE, not JOOL. Also, in my personal opinion, joules are light green.

\ker Yes!

\gr But electricity is definitly blue. Eye melting blue that appears as white, but its definitly blue.

\kev Told ya!

\tek So you're saying that $1J$ is accelerating a kilogram at $1m/s^2$ where you keep applying the force untill it has moved $1m$?

\gr Uh, yeah.

\tek So tell us wiaht somthing is in joules and we will (painfully) convert to kilograms.

\gr One more thing. An equation. $h J=1 Hz*h ?$ where $Hz$ is one $\frac{1}{s}$ or one 'per seccond`. Rearange it to this: $\frac{h J}{1 Hz} = h ?$

\gr $h$ would be in units 'joules per per seccond`. 'Per per` cancels out and you're left with 'joule secconds`

\tek So a joule seccond is\ldots{} $kg*m^2/s$ wich means that\ldots{} the mass, of a surface, per seccond??

\gr $J*s$ isn't a unit ment to mesure things. It only exsits because we multiplied joules and secconds and had to write the units next to it.

\tek Oh, phew.

\gr I'm thinking of a fundemental property of the universe, try to guess waht it is. It is mesured in $J*s$ of course. This constant in particular wants to phrase a joule seccond as being 'energy per frequency`

\tek Energy per frequency. Anyone?

\kev Sound!

\gr Sound can be loud or soft. Different energy.

\ker I was going to say radio, but it can be loud or soft.

\gr Well yes, but, uh, In human numbers we say radio has a spesific energy no matter what.

\tek Not sure the kids will understand but you said your antenna is 50 watts and watts are joules per seccond?

\gr Yeah. Loudness is joules per seccond. One radio particle has a certen number of joules. If you move them, thats joules per seccond, watts. To make it louder, add more particles. More joules per second, more watts.

\ker I get it.

\kev So its not that loud things have big photons, its just there are more of them?

\tek Right, Grace?

\gr Yes. Exactly. We are talking about just one particle.
