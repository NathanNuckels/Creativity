%This is a scrap LaTeX document. It has no header. This is mainly due to the fact that I don't feel like writing one.

\documentclass{article}
\usepackage{amssymb}
\newcommand{\gr}{\noindent$\mathbb{GRACE}$:}
\newcommand{\tek}{\noindent$\mathbb{TEKNIQ}$:}
\newcommand{\kev}{\noindent$\mathbb{KEVIN}$:}
\newcommand{\kem}{\noindent$\mathbb{KERI}$:}
%Commands:

%gr = GRACE
%tek = TEKNIQ
%kev = KEVIN
%kem = KEMS
\begin{document}
\gr Alright class, I gotta talk to you about things that are hard to translate.

\gr Ceasium-133 does\ldots{} somthing nine million times a seccond. You're civilization has atomic clocks, and you define your\ldots{} time thing same as we do.

\gr A meter is one three hundraid millionth of a light seccond. Its impossible to mesure the speed of light but you guys have complex optics (we do too but I don't remember how it works) and can take nanometers and scale them up anthro size with lazers and stuff. I gave you a meter stick, you mesured it, and your definition of a meter is within a tenth of a percent of the real deal and thats good enough for me.

\gr That small percent difference adds up fast, so for large things we use the average distance from Earth to Sol as a unit of mesurement, wich you guys have masured acuratly.

\tek It's not enough for basic stuff. We are still missing a lot of the system though.

\gr I think there's four other units but one of them is way more importent.

\tek Yeah, the kilokam.

\gr Kilogram.

\kev I call it $\mathbb{massthing}$ %Shows up as random math symbols representing the alian language.

\kem Shut up!

I translate his name as Kevin because he absolutly acts like one.

\noindent$\mathbb{randombro}$: Hey, is the dome we're in 37 meters or am I doing it wrong?

\gr GUYS! I wanna talk about killograms!

Everyone stares blankly.

\gr You guys arnt going to like this but-- Just a sec

I check my notes. Dang, this is worse than I thought.

\gr \emph{sigh} Ok, uh, A photon, a radio particle, that ocilates $h^{-1}$ times per seccond has enough energy to accelerate $1kg$ at $1m/s^2$ over the course of $1m$

\kev The heck is that su---

K'emz slaps Kevin because I'm not alloued to.

\gr I said you aint gonna like it! Anyway, you know what $1m/s^2$ would mean, right?

\kev Whats a square seccond in square \emph{reijet}s (Mosphi'ite unit of \emph{length})

\tek Acceleration. Speed go up. Or down. Dosn't matter.

\kev Wait, a seccond mesures time?

\kem Bros got 16KiB of ram.

\gr Literally?

\kem Nah, but his AI is\ldots{} undeveloped.

\kev I mean, I could learn but my machine learning system is $\mathbb{loadabs}$ (Probibly a swear) and makes me go to 784 $\mathbb{notkalvin}$

\tek Grace, thats a temperture unit and it is greater than the boiling point of water by the way.

\kev It makes everyone except me very uncomfortable.

\kem Yeah, you'd boil an organic being alive!

\gr Man, you'd thing a civilization ran by computers would at least have water cooling or somthing--

\tek I use water cooling. The kids don't. It would boil. Noew I'm interested in temperture units.

\gr WE GET THERE WHEN WE GET THERE.

\gr Ok, acceleration. If you want to accelerate an object with mass, what is that called?

\kem Pettal to the metal?

\kev Can I run my machine learning? I promice not to boil anyone.

\tek \emph{sigh} Its $\mathbb{fortis}$ isn't it? Basic stuff.

I edit the translator so whatever TEKNIQ just said becomes 'force`



\gr One newton is accelerating a kilogram at $1m/s^2$. Its force, not energy.

\tek Great. Now you say somthing like ''this thing has a force of X newtons`` and we can calculate a kilogram!

\gr We're not there yet. We need to talk about potential energy.\

\kem And when you do a force over time, thats energy!

\gr No, its---

\tek No no, he's right in his home country. Our equivolent of the metric system does what I think you're about to talk about.

\gr Force, applied over a distance. We call it energy.

\gr Well, newtons per seccond cancels out to killogram meters per seccond so a killogram moving at 1m/s would have 1 ''energy``

\tek Kenetic energy I guess?

\gr Anyway, energy, not kenetic energy, just energy in general is defined by pushing things. When you use $1N$ of force to push somthing $1m$ thats called a Joule.

\kem Hah, I told you energy was green!

\kev No! Its blue! I saw electricity and its blue!

\gr Wha?

\tek Remember, you taught kepler's laws in Kerbal Space Program. They think you're talking about 'Jool` the planet. 

\gr JOULE, not JOOL. Also, in my personal opinion, joules are light green.

\kem Yes!

\gr But electricity is definitly blue. Eye melting blue that appears as white, but its definitly blue.

\kev Told ya!

\tek So you're saying that $1J$ is accelerating a kilogram at $1m/s^2$ where you keep applying the force untill it has moved $1m$?

\gr Uh, yeah.

\tek So tell us wiaht somthing is in joules and we will (painfully) convert to kilograms.

\gr One more thing. An equation. $h J=1 Hz*h ?$ where $Hz$ is one $\frac{1}{s}$ or one 'per seccond`. Rearange it to this: $\frac{h J}{1 Hz} = h ?$

\gr $h$ would be in units 'joules per per seccond`. 'Per per` cancels out and you're left with 'joule secconds`

\tek So a joule seccond is\ldots{} $kg*m^2/s$ wich means that\ldots{} the mass, of a surface, per seccond??

\gr $J*s$ isn't a unit ment to mesure things. It only exsits because we multiplied joules and secconds and had to write the units next to it.

\tek Oh, phew.

\gr I'm thinking of a fundemental property of the universe, try to guess waht it is. It is mesured in $J*s$ of course. This constant in particular wants to phrase a joule seccond as being 'energy per frequency`

\tek Energy per frequency. Anyone?

\kev Sound!

\gr Sound can be loud or soft. Different energy.

\kem I was going to say radio, but it can be loud or soft.

\gr Well yes, but, uh, In human numbers we say radio has a spesific energy no matter what.

\tek Not sure the kids will understand but you said your antenna is 50 watts and watts are joules per seccond?

\gr Yeah. Loudness is joules per seccond. One radio particle has a certen number of joules. If you move them, thats joules per seccond, watts. To make it louder, add more particles. More joules per second, more watts.

\kem I get it.

\kev So its not that loud things have big photons, its just there are more of them?

\tek Right, Grace?

\gr Yes. Exactly. We are talking about just one particle. Not all light has the same energy.

\tekniq Long range comms are low energy. They bounce off the sky and can travel far. Vox passes through buildings. Mirror waves pass through thin stuff, and a
small portion of them can be seen by Grace. Cosmic light

Vox is them communicating with eachother electronicly, like UHF. Mirror is human visible light. Cosmic light is gamma rays.

The reason they have odd names is they are compound words like ''pancakes`` but my translator software I haphasardly whiped up sees ''pan`` and ''cakes`` as seperate words and translates them literally.

\gr Clearly you all know light has frequency. The energy of light is linear. If the frequency doubles, the energy doubles.

\kev So how much energy do they have EXACTICLY?

\gr Thats the definition of a kilogram. The energy contained in light is its frequency multiplied by $h$.

\tek Hey, we use the same equation!

\gr $h$ is different in different units. Its called Planck's constant.

\tek Well, ''Planck's constant`` for us is\ldots{} Well, I don't have it memorized. What's the human Planck constant?

\gr I don't know it in human units.

\kev So we'll never know how much a kilogram is? But then none of your human science equations will work!

\tek Well, I guess after a while of dealing with 'kilograms` we will eventually find a way to convert. A proportion.

\gr Oh, proportions! The definition is all that Planck stuff but before that happened, we defined it as the mass of a cubic decimeter of water.

\tek Cool.

\gr No, not cool. It has to be liquid water.

\tek You guys have solid water!?

\gr Well yeah. Its cold.

\tek The freezing point of water is -58 $\mathbb{notkalvihin}$!!!

\gr Uh, we'll learn temperture tomarrow.

\kev What is ''tomarrow``?

\gr The day after today. Today ends when you go to sleep. Its clearly night time. I'm not tired because I'm from a different timezone. Terran Atomic Time, minus five hours.

\tek I assume time zones are like clock sectors.

\gr Guess your language calls it somthing else. Anyway, tomarrow, we learn temperture.

\kem But I want it now!

\tek Earth tomarrow, Solar tomarrow, or biological tomarrow?

\gr Huh?

\tek Earth tomarrow happens between 12 and 24 hours from now, as you said. Solar tomarrow starts at suntize, biological tomarrow starts when we wake up.

\gr And when do you wake up?

\tek \ldots{} We sleep on a cycle. 10-14 out of every 72 hours conseculativly. You?

\gr between 5 and 7 consecetive hours out of every 24. More is better. We also may sleep for one hour when we should be awake. Always exactlcly one or two hours. Exacticly. Odd isn't it?

\tek same. 2 sets of 2 hours we nap. Not on a cycle, just whenever. Some authoritarian places have a mandated nap time where everyone stops at the same time. Everyone naps at different times and you can't change that. It made me feel more tired back when I was there.

\gr I come from an authoritarian place to. We have a mandated sleep schedual. Some humans are built to be awake during night, and some in day but we're all forced to be awake in the day. Anyway, see you all tomarrow.

\kem Wich one?

\gr Human tomarrow at +8 hours.

\tek Use absolute units, for the kids.

\gr $86400\frac{8}{24}$ is $28800$s.

I wanted to make this send off personal. I converted it to base 4 (13100000) and spoke my final words in thir native language:

\gr \textipa{/re.t'er/ /en/ /ti.mo.ti.zer/ /pev/}

((I'll) return in 1 3 1 0 * 10000)

Their eyes lit up, metaphoricly and literally, as I walked out to sleep. Hopefully the sun- er- Tau will be out when I wake up.

% 1 4 16 64 256 1024 4096 16384 65536 262144
%262144 65536 16384 4096 1024 256 64 16 4 1
%  0       0     1   3     1   0   0  0  0 0
% 81920

\end{document}
